\input{../YKY-preamble.tex}
\setmainfont[BoldFont=AlibabaSans-Regular.otf]{AlibabaSans-Light.otf}

\usepackage[active,tightpage]{preview}		% for continuous page(s)zhong
\renewcommand{\PreviewBorder}{0.5cm}
\renewcommand{\thempfootnote}{\arabic{mpfootnote}}

\usepackage[absolute,overlay]{textpos}		% for page number on upper left corner

\usepackage{color}
\usepackage{mathtools}
\usepackage[hyperfootnotes=false]{hyperref}

% \usepackage[backend=biber,style=numeric]{biblatex}
% \bibliography{../AGI-book}
% \renewcommand*{\bibfont}{\footnotesize}

\usetikzlibrary{shapes}
\usepackage[export]{adjustbox}				% ??
\usepackage{bm}
\usepackage{verbatim} % for comments
% \usepackage{newtxtext,newtxmath}	% Times New Roman font

% \numberwithin{equation}{subsection}

\newcommand{\underdash}[1]{%
	\tikz[baseline=(toUnderline.base)]{
		\node[inner sep=1pt,outer sep=10pt] (toUnderline) {#1};
		\draw[dashed] ([yshift=-0pt]toUnderline.south west) -- ([yshift=-0pt]toUnderline.south east);
	}%
}%

% Tic-Tac-Toe symbols
% \newcommand{\bO}[0]{\raisebox{-0.2em}{\textbf{O}}}
% \newcommand{\Xb}[0]{\raisebox{-0.2em}{\textbf{X}}}

%\DeclareSymbolFont{symbolsC}{U}{txsyc}{m}{n}
%\DeclareMathSymbol{\strictif}{\mathrel}{symbolsC}{74}
\DeclareSymbolFont{AMSb}{U}{msb}{m}{n}
\DeclareSymbolFontAlphabet{\mathbb}{AMSb}
% \setmathfont{Latin Modern Math}

% \newcommand{\highlight}[1]{\colorbox{pink}{$\displaystyle #1$}}

% \newcommand{\emp}[1]{{\color{violet}\textbf{#1}}}
\let\oldtextbf\textbf
\renewcommand{\textbf}[1]{\textcolor{blue}{\oldtextbf{#1}}}

\newcommand*\confoundFace{$\vcenter{\hbox{\includegraphics[scale=0.2]{../2020/../confounded-face.jpg}}}$}
\newcommand{\underconst}{\includegraphics[scale=0.5]{../2020/UnderConst.png}}
\newcommand{\KBsymbol}{\vcenter{\hbox{\includegraphics[scale=1]{../KB-symbol.png}}}}
\newcommand{\witness}{\scalebox{0.6}{$\blacksquare$}}
% \newcommand{\Heytingarrow}{\mathrel{-}\mathrel{\triangleright}}
% \providecommand\Heytingarrow{\relbar\joinrel\mathrel{\vcenter{\hbox{\scalebox{0.75}{$\rhd$}}}}}

\begin{document}

\begin{preview}

\cc{
\title{\vspace{-2.2cm} \bfseries\color{blue}{\large 逻辑化 AGI 基础}}
}{
\title{\vspace{-2.2cm} \bfseries\color{blue}{\large Logicalized AGI basics}}
}

% \author{YKY} % Your name
\date{\vspace{-3.2cm}} % Date, can be changed to a custom date

\maketitle

\setcounter{section}{-1}

% (1) Circled page number on upper left corner
\begin{textblock*}{5cm}(2.1cm,2.3cm) % {block width} (coords) 
{\color{red}{\large \textcircled{\small 1}}}
\end{textblock*}

\begin{minipage}{\textwidth}
\setlength{\parskip}{0.4\baselineskip}

最基本结构是这样:
\begin{equation}
\vcenter{\hbox{\includegraphics[scale=0.5]{../2021/RL-WM-architecture.png}}}
\end{equation}
\vspace{-0.7cm}}:
再详细一点是这样:
\begin{equation}
\vcenter{\hbox{\includegraphics[scale=0.7]{../2021/minimal-RL-architecture.png}}}
\end{equation}

In classical logic-based AI, \textbf{inductive learning} means searching for a \textbf{theory} $T$ (= set of logic rules) that ``explains'' or \textbf{implies} positive examples but not negative ones:
\begin{equation}
\mbox{\large\(
T \vdash e^{+}, \quad T \nvdash e^{-}
\)}
\end{equation}
While logic learning is powerful, it relies on \textbf{combinatorial search} and was too inefficient, which caused ``AI Winter''.

\cc{
我理论的重点是: 透过 $F$ 在 RL 闭环内的训练,可以做到 \textbf{发现} 逻辑理论 $T$ 的效果。 }{
The main thrust of my argument is that training the function $F$ in the RL closed loop can effectively \textbf{discover} the logic theory $T$.}
\end{minipage}
\end{preview}

\begin{preview}
\begin{minipage}{\textwidth}
\setlength{\parskip}{0.4\baselineskip}	

\begin{textblock*}{20cm}(2.1cm,2cm) % {block width} (coords) 
	{\color{red}{\large \textcircled{\small 2}}}
	\hspace{8cm}
	\color{blue}{\footnotesize \cc{逻辑化 AGI 基础}{Logicalized AGI}}
\end{textblock*}

\vspace*{0.3cm} 
\cc{
首先可以了解一下几种``机器''之间的关系。 图灵机 演变成 \textbf{神经图灵机},这是``\textbf{注意力机制}''的起源。 \textbf{自注意力} 的特点是它有 \textbf{equivariance} 这种对称性,亦即是说,输入/输出元素的\textbf{次序}不重要。 也可以将这些元素看成一个 \textbf{集合} (例如 $\{1,2,3\}$ 跟 $\{3,2,1\}$ 是同一个集合)。 这种结构适合处理 \textbf{逻辑命题},因为 $A \wedge B = B \wedge A$.}{
Let's look at the interconnections between some ``learning machines''.  Turing Machines inspired \textbf{Neural Turing Machines}, which is the origin of the \textbf{Attention} mechanism.  \textbf{Self-Attention} has the symmetry of \textbf{equivariance}, which means that the \textbf{order} of input / output elements is unimportant.  Another way to put it is that such elements have a \textbf{set structure}; for example $\{1,2,3\}$ and $\{3,2,1\}$ are the same set.  The set structure is also suitable for handling logic propositions, because $A \wedge B = B \wedge A$.
}

\cc{
\textbf{Graph} 也是一种逻辑结构,因为 graph 可以\textbf{分解}为一堆节点之间的关系,例如 张三是李四的朋友 $\Rightarrow$ friend(Zhang$_3$, Li$_4$). 所以 \textbf{GNN} 是一种逻辑处理器。 另方面,语言模型 也使用 \textbf{Transformer} / Self-Attention. 所以我们推测,在 Transformer 语言模型里 也有逻辑规则的\textbf{涌现} (emergence), 而 Transformer circuits 的研究部分地证实了这一想法。}{
A \textbf{graph} is also a logical structure, as it can be decomposed into a bunch of nodes and links (relations).  For example, John is Pete's friend $\Rightarrow$ friend(John, Pete).  Thus the \textbf{GNN} is a logic processor.  Similarly, Language Models are built from \textbf{Transformers} / Self-Attention.  So we believe that logic-like rules may \textbf{emerge} in Transformer-based language models.  Research on ``Transformer circuits'' partially confirms this.
}

\begin{equation}
\vcenter{\hbox{\includegraphics[scale=0.8]{AGI-standard-model.png}}}
\end{equation}
\cc{
然后关键的一步就是将 \textbf{end-to-end} 训练的模型 变成 \textbf{RL} 训练的形式\footnote{通常 RL 的 actions 是在环境中的动作,但我的 RL 模型 要求 actions 是一些 \textbf{思想} (thoughts),这并不规范,可能产生传统 RL 没有的问题。}。
前者可以看成是 进行 ``few-steps'' 的逻辑推导,而后者是可以 ``任意多步'' 的推导,而且这些推导 互相之间有 \textbf{协同效应} (一个推出来的结果帮助另一个新的推导)。 正是因为 \textbf{闭环训练} 容许了这些 协同效应,令 RL 系统有可能学习出能够\textbf{解释}世界的 logic theory.
}{
The next, crucial step is to change end-to-end training to the RL setting\footnote{Normally, actions in RL refer to what are performed in the external environment, but in my somewhat unorthodox formulation, actions are ``thoughts''.  This may create some problems not seen in traditional RL.}.
The former can be seen as allowing ``few-steps'' logic inference, whereas the latter allows \textbf{arbitrary number} of inference steps.  This enables logic rules to make use of results from other rules, creating a \textbf{cooperative effect}.  Precisely due to closed-loop training and the synergy, an RL agent may be able to learn a logic theory that \textbf{explains} the world.
}
\end{minipage}
\end{preview}

\begin{preview}
	
\setlength{\parskip}{0.4\baselineskip}
\begin{textblock*}{20cm}(2.1cm,2cm) % {block width} (coords) 
	{\color{red}{\large \textcircled{\small 3}}}
	\hspace{8cm}
	\color{blue}{\footnotesize 逻辑化 AGI 基础}
\end{textblock*}
\vspace*{0.3cm} 

\end{preview}
\end{document}
