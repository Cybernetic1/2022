\input{../YKY-preamble.tex}
\setmainfont[BoldFont=AlibabaSans-Regular.otf,ItalicFont=AlibabaSans-LightItalic.otf]{AlibabaSans-Light.otf}
	
\usepackage[active,tightpage]{preview}		% for continuous page(s)
\renewcommand{\PreviewBorder}{0.5cm}
\renewcommand{\thempfootnote}{\arabic{mpfootnote}}

\usepackage[absolute,overlay]{textpos}		% for page number on upper left corner

\usepackage{color}
\usepackage{mathtools}
\usepackage[hyperfootnotes=false]{hyperref}

% \usepackage[backend=biber,style=numeric]{biblatex}
% \bibliography{../AGI-book}
% \renewcommand*{\bibfont}{\footnotesize}

\usetikzlibrary{shapes}
\usepackage[export]{adjustbox}				% ??
\usepackage{bm}
\usepackage{verbatim} % for comments
% \usepackage{newtxtext,newtxmath}	% Times New Roman font

% \numberwithin{equation}{subsection}

\newcommand{\underdash}[1]{%
	\tikz[baseline=(toUnderline.base)]{
		\node[inner sep=1pt,outer sep=10pt] (toUnderline) {#1};
		\draw[dashed] ([yshift=-0pt]toUnderline.south west) -- ([yshift=-0pt]toUnderline.south east);
	}%
}%

% Tic-Tac-Toe symbols
% \newcommand{\bO}[0]{\raisebox{-0.2em}{\textbf{O}}}
% \newcommand{\Xb}[0]{\raisebox{-0.2em}{\textbf{X}}}

%\DeclareSymbolFont{symbolsC}{U}{txsyc}{m}{n}
%\DeclareMathSymbol{\strictif}{\mathrel}{symbolsC}{74}
\DeclareSymbolFont{AMSb}{U}{msb}{m}{n}
\DeclareSymbolFontAlphabet{\mathbb}{AMSb}
% \setmathfont{Latin Modern Math}

% \newcommand{\highlight}[1]{\colorbox{pink}{$\displaystyle #1$}}

% \newcommand{\emp}[1]{{\color{violet}\textbf{#1}}}
\let\oldtextbf\textbf
\renewcommand{\textbf}[1]{\textcolor{blue}{\oldtextbf{#1}}}

\newcommand*\smileFace{$\vcenter{\hbox{\includegraphics[scale=0.6]{../smiley.jpg}}}$}
\newcommand{\underconst}{\includegraphics[scale=0.5]{../2020/UnderConst.png}}
\newcommand{\KBsymbol}{\vcenter{\hbox{\includegraphics[scale=1]{../KB-symbol.png}}}}
\newcommand{\witness}{\scalebox{0.6}{$\blacksquare$}}
% \newcommand{\Heytingarrow}{\mathrel{-}\mathrel{\triangleright}}
% \providecommand\Heytingarrow{\relbar\joinrel\mathrel{\vcenter{\hbox{\scalebox{0.75}{$\rhd$}}}}}

\begin{document}

\begin{preview}

\cc{
\title{\vspace{-1.5cm} \bfseries\color{blue}{\Large AGI 理论撮要}}
}{
\title{\vspace{-1.5cm} \bfseries\color{blue}{\Large AGI theory summary}}
}

% \author{YKY} % Your name
\date{\vspace{-2cm}} % Date, can be changed to a custom date

\maketitle

\setcounter{section}{-1}

% (1) Circled page number on upper left corner
\begin{textblock*}{5cm}(2.1cm,2.3cm) % {block width} (coords) 
{\color{red}{\large \textcircled{\small 1}}}
\end{textblock*}

\begin{minipage}{\textwidth}
\setlength{\parskip}{0.4\baselineskip}

很多人不熟悉 经典逻辑 AI,可以参考 AIMA\footnote{}.  简单来说,

最简单的 智能系统架构 是这样:



\end{minipage}
\end{preview}

\begin{preview}
\begin{minipage}{\textwidth}
\setlength{\parskip}{0.4\baselineskip}

\begin{textblock*}{20cm}(2.1cm,2cm) % {block width} (coords) 
	{\color{red}{\large \textcircled{\small 2}}}
	\hspace{8cm}
	\color{blue}{\footnotesize \cc{Transformer 的逻辑解释}{Transformer as Logic}}
\end{textblock*}

\vspace*{0.3cm} 

\end{minipage}
\end{preview}

\begin{preview}
\begin{minipage}{\textwidth}
	
\setlength{\parskip}{0.4\baselineskip}
\begin{textblock*}{20cm}(2.1cm,2cm) % {block width} (coords) 
	{\color{red}{\large \textcircled{\small 3}}}
	\hspace{8cm}
	\color{blue}{\footnotesize \cc{Transformer 的逻辑解释}{Transformer as Logic}}
\end{textblock*}

\vspace*{0.3cm} 


\end{minipage}
\end{preview}

\begin{preview}
\begin{minipage}{\textwidth}

\setlength{\parskip}{0.4\baselineskip}
\begin{textblock*}{20cm}(2.1cm,2cm) % {block width} (coords) 
	{\color{red}{\large \textcircled{\small 4}}}
	\hspace{8cm}
	\color{blue}{\footnotesize \cc{Transformer 的逻辑解释}{Transformer as Logic}}
\end{textblock*}

\vspace*{0.3cm} 


\end{minipage}
\end{preview}
\end{document}
