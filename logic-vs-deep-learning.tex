\input{../YKY-preamble.tex}
\setmainfont[BoldFont=Alibaba_Sans_Regular.otf,ItalicFont=Alibaba_Sans_Light_Italic.otf]{Alibaba_Sans_Light.otf}
	
\usepackage[active,tightpage]{preview}		% for continuous page(s)
\renewcommand{\PreviewBorder}{0.5cm}
\renewcommand{\thempfootnote}{\arabic{mpfootnote}}

\usepackage[absolute,overlay]{textpos}		% for page number on upper left corner

\usepackage{color}
\usepackage{mathtools}
\usepackage[hyperfootnotes=false]{hyperref}

% \usepackage[backend=biber,style=numeric]{biblatex}
% \bibliography{../AGI-book}
% \renewcommand*{\bibfont}{\footnotesize}

\usetikzlibrary{shapes}
\usepackage[export]{adjustbox}				% ??
\usepackage{verbatim} % for comments
% \usepackage{newtxtext,newtxmath}	% Times New Roman font

% \numberwithin{equation}{subsection}

\newcommand{\underdash}[1]{%
	\tikz[baseline=(toUnderline.base)]{
		\node[inner sep=1pt,outer sep=10pt] (toUnderline) {#1};
		\draw[dashed] ([yshift=-0pt]toUnderline.south west) -- ([yshift=-0pt]toUnderline.south east);
	}%
}%


%\DeclareSymbolFont{symbolsC}{U}{txsyc}{m}{n}
%\DeclareMathSymbol{\strictif}{\mathrel}{symbolsC}{74}
\DeclareSymbolFont{AMSb}{U}{msb}{m}{n}
\DeclareSymbolFontAlphabet{\mathbb}{AMSb}
% \setmathfont{Latin Modern Math}

% \usepackage[most]{tcolorbox}
\tcbset{on line, 
	boxsep=4pt, left=0pt,right=0pt,top=0pt,bottom=0pt,
	colframe=red,colback=pink,
	highlight math style={enhanced}
}
\newcommand{\atom}{\vcenter{\hbox{\tcbox{....}}}}

% \newcommand{\emp}[1]{{\color{violet}\textbf{#1}}}
\let\oldtextbf\textbf
\renewcommand{\textbf}[1]{\textcolor{blue}{\oldtextbf{#1}}}

\newcommand{\logic}[1]{{\color{violet}{\textit{#1}}}}
\newcommand*\smileFace{$\vcenter{\hbox{\includegraphics[scale=0.6]{../smiley.jpg}}}$}
\newcommand{\underconst}{\includegraphics[scale=0.5]{../2020/UnderConst.png}}
\newcommand{\KBsymbol}{\vcenter{\hbox{\includegraphics[scale=1]{../KB-symbol.png}}}}
\newcommand{\witness}{\scalebox{0.6}{$\blacksquare$}}
% \newcommand{\Heytingarrow}{\mathrel{-}\mathrel{\triangleright}}
% \providecommand\Heytingarrow{\relbar\joinrel\mathrel{\vcenter{\hbox{\scalebox{0.75}{$\rhd$}}}}}

\begin{document}

\begin{preview}

\cc{
\title{\vspace{-1.5cm} \bfseries\color{blue}{\Large 逻辑与深度学习的关系}}
}{
\title{\vspace{-1.5cm} \bfseries\color{blue}{\Large Comparison of Logic AI and Deep Learning}}
}

% \author{YKY} % Your name
\date{\vspace{-2cm}} % Date, can be changed to a custom date

\maketitle

\setcounter{section}{-1}

% (1) Circled page number on upper left corner
\begin{textblock*}{5cm}(2.1cm,2.3cm) % {block width} (coords) 
{\color{red}{\large \textcircled{\small 1}}}
\end{textblock*}

\begin{minipage}{\textwidth}
\setlength{\parskip}{0.4\baselineskip}

这是经典逻辑 AI 的最基本运作模式:
\begin{equation}
\vcenter{\hbox{\includegraphics[scale=1]{LBAI-basic-config.png}}}
\end{equation}

它其实包含了两个算法:
\begin{itemize}
	\item \textbf{matching} (unification): \\
	逻辑 rules 是包含变量的条件命题, \\
	例如 \tab \logic{$\forall x. \mbox{是人}(x) \Rightarrow \mbox{会死}(x). $ }\\
	Unification 判定一条 rule 是否可以 apply 到某逻辑命题上,\\
	例如:\logic{是人(苏格拉底)} 可以跟上式的左边 unify. \\
	Matching 的结果是得到一推 instantiated(特例化,即不包含变量)的命题。
	
	\item \textbf{forward- or backward-chaining} (resolution): \\
	由已知事实 推导出新结论,或反过来,判断某给定的新结论是否成立。 \\
	例如:\logic{ 是人(苏格拉底) $\Rightarrow$ 会死(苏格拉底) $\;\; \wedge$ 是人(苏格拉底) } \\
	可以推出:\logic{会死(苏格拉底)}。
\end{itemize}

深度学习的特点,就是将
\begin{equation}
\mbox{状态}_t  \vdash \mbox{状态}_{t+1}
\end{equation}
的逻辑推导过程,通通纳入进去一个非常复杂的非线性函数(= 深度神经网络)里面。 这样做以后,上述的逻辑结构被
``mingled'' 在一起,以至于很难分辨了。 但也正是由于这种「大杂烩」,深度神经网络 将一套复杂的组合算法 压缩成数量不算太多的一层层的参数。 它同时可以做 learning 和 inference 这两个动作。 这种简单粗暴的方法,其实非常有效率,要超越它的速度并不容易!

我们知道(或推测)一个智能系统 应该具有 符号逻辑的结构。 这点知识可不可以用来 约束/加速 深度神经网络? 答案似乎是有可能的。 现时 state-of-the-art 处理 视觉的 CNN 和 处理文字的 BERT,它们都有内部结构, \textbf{而不是 fully-connected},而且 这内部结构 对应于 被处理的资料的结构。 因此我们有理由相信,逻辑结构 可以用来约束 深度神经网络的结构,达到加速。 

\end{minipage}
\end{preview}

\begin{preview}
\begin{minipage}{\textwidth}
\setlength{\parskip}{0.4\baselineskip}

\begin{textblock*}{20cm}(2.1cm,2cm) % {block width} (coords) 
	{\color{red}{\large \textcircled{\small 2}}}
	\hspace{8cm}
	\color{blue}{\footnotesize \cc{逻辑与深度学习}{Logic and Deep Learning}}
\end{textblock*}
\vspace*{0.3cm} 

接下来我们详细一点看逻辑系统的结构:

Knowledge Base 里面有很多 rules,系统要将这些 rules 逐一 match with 系统状态 (= working memory) 里面的命题:
\begin{equation}
\vcenter{\hbox{\includegraphics[scale=1]{rete-explained-1.png}}}
\end{equation}
成功 matched 的 rules 可以导出新的结论,加进 working memory 的状态 里面。

这个复杂的操作,完全被一个神经网络取代。 或者可以更抽象地说:
\begin{equation}
\label{eqn:some-kind-of-memory}
\vcenter{\hbox{\includegraphics[scale=1]{some-kind-of-memory.png}}}
\end{equation}

而以 Transformer 来说,它是一种 输入元 \textbf{之间} 的记忆体(这记忆就储存在 Q, K, V 矩阵里),而它 \textbf{implicitly} 做到了 rules 的作用:
\begin{equation}
\vcenter{\hbox{\includegraphics[scale=1]{rete-explained-3b.png}}}
\end{equation}
换句话说,Transformer 内部有某种(扭曲了的)逻辑 rules 的结构。 那么很自然的问题就是:能否发掘更多 逻辑/逻辑系统 的结构? 也就是说,公式 (\ref{eqn:some-kind-of-memory}) 可以有怎样的代数结构约束? 这个问题 可以参考 范畴逻辑 的理论,还有 经典 logic-based AI 系统的理论。

\end{minipage}
\end{preview}

\begin{preview}
\begin{minipage}{\textwidth}
	
\setlength{\parskip}{0.4\baselineskip}
\begin{textblock*}{20cm}(2.1cm,2cm) % {block width} (coords) 
	{\color{red}{\large \textcircled{\small 3}}}
	\hspace{8cm}
	\color{blue}{\footnotesize \cc{逻辑与深度学习}{Logic and Deep Learning}}
\end{textblock*}
\vspace*{0.3cm} 

我们希望 勾画出公式 (\ref{eqn:some-kind-of-memory}) 需要具备的代数约束,但暂时先用文字描述比较容易:

\begin{itemize}
	\item 状态是 \textbf{颗粒化} 的,它是某集合的元素,元素之间可交换,也就是 Transformer 的 equivariance. (注意: Transformer 有 equivariance,但 equivariance 未必一定要用 Transformer 实现)

	\item \textbf{深度结构}:例如多层网络,每层是函数的复合 (composition). Transformer 也用了深度结构。
	
	\item 逻辑 包括了 \textbf{命题}层次 和 \textbf{命题内部}层次 的 颗粒化。 后者是\textbf{谓词} (predicate) 逻辑的结构,例如: \logic{loves( John, Mary )},也可以简单地将它视为 \textbf{代数元}之间的\textbf{乘积},例如: \logic{John \bullet\  loves \bullet\   Mary}, 后者也叫做 ``word''.  (不同类别的代数元之间不一定容许乘积,因此有 groupoid 的概念,但暂时来说这细节不重要。) 现时重点是如何将 这两层的 颗粒化 结构 施加到深度神经网络上。 

	\item 逻辑推导 每步只产生\textbf{一个}新的结论(或其概率分布),然后这个新的结论,再加入到旧的状态中,作为一个命题集合的元素,而旧状态也要 \textbf{遗忘} 一些命题,否则需要无限记忆。 这跟 Transformer 每次输出\textbf{一列}的 tokens 有点不同(虽然我们不太肯定 Transformer tokens 究竟对应于 命题 还是 谓词/原子概念)。 
	
	\item 逻辑 rule 通常只跟某几个前提有关,其它前提是\textbf{无关}的,例如: \logic{眼睛好看 $\wedge$ 鼻子好看 $\wedge$ 嘴巴好看 $\Rightarrow$ 帅},跟 \logic{有钱} 或 \logic{穷} 无关。 Transformer 的 \textbf{softmax} 结构似乎也可以排除一些无关的 tokens 的影响。

	\item (可能还有其他的结构特征.....)
\end{itemize}


\end{minipage}
\end{preview}

\begin{preview}
\begin{minipage}{\textwidth}

\setlength{\parskip}{0.4\baselineskip}
\begin{textblock*}{20cm}(2.1cm,2cm) % {block width} (coords) 
	{\color{red}{\large \textcircled{\small 4}}}
	\hspace{8cm}
	\color{blue}{\footnotesize \cc{逻辑与深度学习}{Logic and Deep Learning}}
\end{textblock*}
\vspace*{0.3cm} 

根据我的理论,理想的逻辑形式是这样的(各种元素的数量纯粹示意):
\begin{equation}
\label{fig:logic-symmetry-predicate-level}
\vcenter{\hbox{\includegraphics[scale=1]{logic-symmetry-predicate-level.png}}}
\end{equation}

相比之下,我们目前可以写出来的 代数关系 $p \wedge q = p \wedge q$ 只表达了这种结构:
\begin{equation}
\label{fig:logic-symmetry-propositional-only}
\vcenter{\hbox{\includegraphics[scale=1]{logic-symmetry-propositional-only.png}}}
\end{equation}
相比于 图 (\ref{fig:logic-symmetry-propositional-only}),图 (\ref{fig:logic-symmetry-predicate-level}) 添加了 $\atom$ 的约束,但这约束怎样用代数表示?

而 Transformer 的结构更像这样:
\begin{equation}
\label{fig:logic-symmetry-Transformer}
\vcenter{\hbox{\includegraphics[scale=1]{logic-symmetry-Transformer.png}}}
\end{equation}
它没有 explicit 的命题结构,而是用特别的 token 表示句子的完结,当然还有 positional encoding 这些「伎俩」。 所以,Transformer 是一种比较 \textit{ad hoc} 的设计,我们应该可以改进它。

但由于 $\atom \in$ 非交换自由群(最少结构的群),它没有像 $a \cdot b = b \cdot a$ 那样的对称性公式。

似乎要考虑  $\atom$ 的群结构是如何影响 逻辑 rule (\ref{fig:logic-symmetry-predicate-level}) 的输入和输出? 类似这样:
\begin{equation}
q_1 \cdot q_2 \; \Longleftarrow \; p_{11} \cdot p_{12} \; \wedge \; p_{21} \cdot p_{22}
\end{equation}
它跟这样有什么不同:
\begin{equation}
Q \; \Longleftarrow \; P_1 \; \wedge \; P_2 \quad ?
\end{equation}

$Q$ 是由两件东西组成,但这两件东西可以换成另外两件。 但那就不再是同一个 rule.  如果要维持是同一个 rule,则 任何 $\atom$ 元素不可以交换或更换。 那是不是需要考虑所有的 rules?

或者可以考虑一个 $\forall$ rule,它包含很多个 instantiated rules.  而它的 substitutions 总是以 群元素为 unit.  这一点可以怎样表示?

但其实当 rule 改变时,它似乎也是根据群元素的边界而改变的。 似乎一定要涉及大量的 rules 才可以表达 $\atom$ 的对称性。

但从机器学习的角度来看,rules 的更新 似乎没有必要 遵从 $\atom$ 的边界?

从经典 AI 的角度看,lattice of rules 结构 就是 algebra of atomic concepts 的结构。

\end{minipage}
\end{preview}

\begin{preview}
\begin{minipage}{\textwidth}

\setlength{\parskip}{0.4\baselineskip}
\begin{textblock*}{20cm}(2.1cm,2cm) % {block width} (coords) 
	{\color{red}{\large \textcircled{\small 5}}}
	\hspace{8cm}
	\color{blue}{\footnotesize \cc{逻辑与深度学习}{Logic and Deep Learning}}
\end{textblock*}
\vspace*{0.3cm} 

%我们希望 以代数形式表达 谓词 $\atom$ 的颗粒性。 
在 范畴逻辑 里面有 \textbf{Beck-Chevalley} 条件和 \textbf{Frobenius} 条件,或许是我们所需的对称性? 但细看之后,发觉还是不能解决问题.....  For completeness,我还是描述一下,没兴趣的可以略过。

首先考虑比较容易明白的 \textbf{Frobenius} 条件。 在逻辑上,它等于说:
\begin{equation}
\exists x. [ \phi \wedge \psi(x) ] \equiv \phi \wedge \exists x. \psi(x).
\end{equation}
由于 经典逻辑 AI 普遍使用 $\forall$ 而忽略 $\exists$,我将上式改写成:
\begin{equation}
\label{eqn:Frobenius-condition}
\forall x. [ \phi \vee \psi(x) ] \equiv \phi \vee \forall x. \psi(x).
\end{equation}
但问题是,(\ref{eqn:Frobenius-condition}) 式的左边和右边,其对应的神经网络 (\ref{fig:logic-symmetry-predicate-level}) 是一样的(看不出有分别)。 也就是说这个差别可能太 subtle 了,它并不影响我们实际 implement 的神经网络。

在逻辑里,任何变量 例如 $x,y$ 等,必须被 $\forall$ 或 $\exists$ quantify,否则不成为合法的句子。 所以 表达 谓词结构的对称性,也很可能要涉及 $\forall$ 或 $\exists$.

\end{minipage}
\end{preview}

\begin{preview}
\begin{minipage}{\textwidth}
		
\setlength{\parskip}{0.4\baselineskip}
\begin{textblock*}{20cm}(2.1cm,2cm) % {block width} (coords) 
	{\color{red}{\large \textcircled{\small 6}}}
	\hspace{8cm}
	\color{blue}{\footnotesize \cc{逻辑与深度学习}{Logic and Deep Learning}}
\end{textblock*}
\vspace*{0.3cm} 

以前说过,谓词逻辑 带来 \textbf{fibration} 或 \textbf{indexing} 结构。 Beck-Chevalley 和 Frobenius 条件 基本上是说,这 纤维结构 是 ``preserved by re-indexing functors''.

这是 fibration 结构的示意图:
\begin{equation}
\vcenter{\hbox{\includegraphics[scale=0.7]{etale-space.png}}}
\end{equation}
这整个结构 叫 \textbf{bundle},而 \textbf{sheaf} 是 bundle 加上某个特殊的 拓扑结构。 

在 $(A,f)$ 和 $(B,g)$ 两个 bundle 之上可以定义 \textbf{fibred product} of $A$ and $B$ over $I$, 记作 $A \times_I B$:
\begin{equation}
\label{eqn:fibred-product}
\begin{tikzcd}[sep = 3cm]
A \times B \arrow[r, "q"] \arrow[d, swap, "p"] \arrow[rd, "h"] & B \arrow[d, "g"] \\
A \arrow[r, swap, "f"] & I
\end{tikzcd}
\end{equation}
其中 $h = f \circ p = g \circ q$.  这也是一个 \textbf{pullback}.

\textbf{Beck-Chevalley} 条件是说 下面这幅图 commute: 
\begin{equation}
\label{eqn:Beck-Chevalley}
\begin{tikzcd}[column sep = 3cm]
K \times J \arrow[r, "u \;\times\; id"] \arrow[d, swap, "\pi"] & I \times J \arrow[d, "\pi"] \\
K \arrow[r, swap, "u"] & I
\end{tikzcd}
\end{equation}
其中 $\pi$ 就是代表 量词 $\forall$ 或 $\exists$ 的 投影,它们是 weakening map $\pi^*$ 的伴随映射。 

Beck-Chevalley 条件并不完全是空洞的;它有可能不成立。 有一个反例是 Pitts 提出的: 考虑 $X \times Y$, 其中 $X = Y = \mathbb{N} \cup \{\infty\}$ 亦即 自然数加上 $\infty$ 作为 top element; 但 $Y$ 是用 discrete order,亦即所有 order 都是 =.  $A$ 是 $X \times Y$ 上的关系: $A = \{ (x,y) \in \mathbb{N \times N} \;|\; x \le y \}$.  那么 $\exists y. (x,y) \in A$ 会是整个 $X$ 集合。 如果考虑 DCPO 范畴,我们要求 fibration of Scott-closed subsets (ordered by inclusion) over DCPO.  $\exists y. A$ 的 Scott closure 的条件是 它是一个 lower set closed under directed joins; 而这个 Scott closure 条件似乎不成立,因而导致 图 (\ref{eqn:Beck-Chevalley}) 不 commute.(我对 Scott closure 的细节不太理解)

Lawvere 的工作将 Beck-Chevalley 条件变得更一般化: 「简单」的 $\forall$ 和 $\exists$ 量词 是 weakening functor $\pi^*$(基于笛卡尔积)的伴随函子,但 Lawvere 将它扩充到任何 \textbf{substitution} functor $u^*$.

\end{minipage}
\end{preview}

\end{document}
