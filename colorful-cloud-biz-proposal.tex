\input{../YKY-preamble.tex}
\setmainfont[BoldFont=Alibaba_Sans_Regular.otf,ItalicFont=Alibaba_Sans_Light_Italic.otf]{Alibaba_Sans_Light.otf}
	
\usepackage[active,tightpage]{preview}		% for continuous page(s)
\renewcommand{\PreviewBorder}{0.5cm}
\renewcommand{\thempfootnote}{\arabic{mpfootnote}}

\usepackage[absolute,overlay]{textpos}		% for page number on upper left corner

\usepackage{color}
\usepackage{mathtools}
\usepackage[hyperfootnotes=false]{hyperref}

% \usepackage[backend=biber,style=numeric]{biblatex}
% \bibliography{../AGI-book}
% \renewcommand*{\bibfont}{\footnotesize}

\usetikzlibrary{shapes}
\usepackage[export]{adjustbox}				% ??
\usepackage{verbatim} % for comments
% \usepackage{newtxtext,newtxmath}	% Times New Roman font

\titleformat{\subsection}[hang]{\bfseries\large\color{blue}}{}{0pt}{} 
% \numberwithin{equation}{subsection}

\newcommand{\underdash}[1]{%
	\tikz[baseline=(toUnderline.base)]{
		\node[inner sep=1pt,outer sep=10pt] (toUnderline) {#1};
		\draw[dashed] ([yshift=-0pt]toUnderline.south west) -- ([yshift=-0pt]toUnderline.south east);
	}%
}%


%\DeclareSymbolFont{symbolsC}{U}{txsyc}{m}{n}
%\DeclareMathSymbol{\strictif}{\mathrel}{symbolsC}{74}
\DeclareSymbolFont{AMSb}{U}{msb}{m}{n}
\DeclareSymbolFontAlphabet{\mathbb}{AMSb}
% \setmathfont{Latin Modern Math}
\DeclareMathOperator*{\argmin}{arg\,min}

% \usepackage[most]{tcolorbox}
\tcbset{on line, 
	boxsep=4pt, left=0pt,right=0pt,top=0pt,bottom=0pt,
	colframe=red,colback=pink,
	highlight math style={enhanced}
}
\newcommand{\atom}{\vcenter{\hbox{\tcbox{....}}}}

\let\oldtextbf\textbf
\renewcommand{\textbf}[1]{\textcolor{blue}{\oldtextbf{#1}}}

\newcommand{\logic}[1]{{\color{violet}{\textit{#1}}}}
\newcommand{\underconst}{\includegraphics[scale=0.5]{../2020/UnderConst.png}}
\newcommand{\KBsymbol}{\vcenter{\hbox{\includegraphics[scale=1]{../KB-symbol.png}}}}
\newcommand{\token}{\vcenter{\hbox{\includegraphics[scale=1]{token.png}}}}
\newcommand{\proposition}{\vcenter{\hbox{\includegraphics[scale=0.8]{proposition.png}}}}

\begin{document}

\begin{preview}

\cc{
\title{\vspace{-1.5cm} \bfseries\color{blue}{\LARGE 彩云科技 商业建议书}}
}{
\title{\vspace{-1.5cm} \bfseries\color{blue}{\LARGE Colorful Cloud business proposal}}
}

% \author{YKY} % Your name
\date{\vspace{-2cm}} % Date, can be changed to a custom date

\maketitle

\setcounter{section}{-1}

% (1) Circled page number on upper left corner
\begin{textblock*}{5cm}(2.1cm,2.3cm) % {block width} (coords) 
{\color{red}{\large \textcircled{\small 1}}}
\end{textblock*}

\begin{minipage}{\textwidth}
\setlength{\parskip}{0.4\baselineskip}

Dr 肖达、袁行远,您们好:

彩云科技 提供以下:
\begin{itemize}
	\item 租用 GPT 服务器/硬件
	\item 使用 GPT 方面的技术支缓
	\item 分享模型,合作研究
	\item 股份交换 (optional)
\end{itemize}

我们 DAO 提供以下:
\begin{itemize}
	\item 一次性 合作费 \$$F$ = ¥10万-100万
	\item 一次性 股份 $S$\% = 40-50\%
	\item 每月 租金 \$$M$ = ¥1万(一年后续约)
	\item 每月 我们公司赚得利润的 $p$\% = 30-50\%(一年后续约)
\end{itemize}

长远来说,我们公司的 业务和 模型 会有所变化,一年后:
\begin{itemize}
	\item 租金 \$$M$ 重新决定
	\item 利润 $p$\% 重新决定
	\item 我们 DAO 股份 每半年 dilute 一次
\end{itemize}

\end{minipage}
\end{preview}

\begin{comment}
\begin{preview}
\begin{minipage}{\textwidth}
\setlength{\parskip}{0.4\baselineskip}

\begin{textblock*}{20cm}(2.1cm,2cm) % {block width} (coords) 
	{\color{red}{\large \textcircled{\small 2}}}
	\hspace{8cm}
	\color{blue}{\footnotesize \cc{逻辑 Transformer}{Logic Transformer}}
\end{textblock*}
\vspace*{0.3cm} 

\end{minipage}
\end{preview}
\end{comment}

\end{document}
